\chapter*{Úvod}

\textit{Martin \v Sm\'\i d, BISOP}
\vspace{15mm}

Když začne hořet dům, každý jde hasit. Na počátku pandemie tak mělo mnoho lidí potřebu nějak se situací pomoci. Někdo šil roušky, někdo pomáhal izolovaným seniorům, někdo se snažil sbírat a zveřejňovat data. Přesto málokoho z vědců ne-epidemiologů napadlo, že by jejich angažmá mohlo být užitečné. Vláda má přece jistě dost svých odborníků a sofistikovaných modelů! Když se však začalo proslýchat, že odborníkem, který přesvědčil vládu k rychlé reakci, byl manažer pojišťovny Pavel Řehák, a to za pomoci excelovské tabulky, začalo být jasné, že jakákoli nevládní aktivita je namístě.

Během prvních týdnů pandemie vzniklo několik vědeckých iniciativ, mezi nimi i iniciativa {\em Model antiCOVID-19 pro ČR}, která si dala za cíl tvorbu realistických modelů šíření epidemie. Postupně začaly vznikat tři takové modely: globální Model A, podrobnější Model B a tzv. agentní model, který dostal označení M, protože popisoval šíření epidemie ve středně velkém městě. Hlavní ambicí iniciativy bylo využití modelů a jejich predikcí jako podpory rozhodování o protiepidemických opatřeních. 

Pod vedením Reného Levínského a Josefa Šlerky se shromáždil malý tým, většina jehož členů měla zkušenosti s matematickým modelováním. Všichni si uvědomovali, že vstupují do neznámých vod, nicméně tehdejší atmosféra, která podobným iniciativám přála, je ke vstupu na neznámou půdu motivovala. První verze modelů vznikly na vědecké poměry velice rychle, takže mohly být představeny už koncem května na konferenci NZIS Open\footnote{\url{https://www.uzis.cz/index.php?pg=aktuality&aid=8399}}. Tam se mimo jiné ukázalo, že skupina ve svém úsilí není osamocena: své nově vyvinuté modely tu představila jak skupina Lenky Přibylové z Masarykovy univerzity, tak státní Ústav zdravotnických informací a statistiky. 

Nezávisle na modelovacím týmu se v prvních týdnech pandemie zformovala skupina dobrovolníků okolo zmíněného Pavla Řeháka, Pavla Hroboně, Jakuba Drbohlava a Evy Blechové, snažící se vládě pomoci zavést takzvanou chytrou karanténu, zejména nastavit efektivní systém trasování rizikových kontaktů. Na konci května, kdy se mnohým zdálo, že epidemie končí, zformulovala skupina materiál {\em Systém rychlé reakce na případnou druhou vlnu covid-19 a další epidemie}\footnote{\url{https://texty.hlidacstatu.cz/system-rychle-reakce-na-pripadnou-druhou-vlnu-covid-19-a-dalsi-epidemie/}}, ve kterém navrhla 32 kroků, které měly zemi připravit na očekávanou podzimní vlnu. 

Spojením „trasovací“ a „modelovací“ skupiny pak v létě 2020 vzniklo Centrum pro modelování biologických a společenských procesů (BISOP). Formálně jde o zapsaný ústav, prakticky o platformu zastřešující spolupráci skupiny, jejíž členové 
se účastní buď jako dobrovolníci, nebo spolupráci spojují s aktivitami ve svých domovských institucích. 

Hned po svém vzniku se Centrum pro modelování biologických a společenských procesů zapojilo do celospolečenské diskuse, ať již se jednalo o popularizaci vědeckého přístupu k pandemii či o konkrétní otázky týkající se vývoje epidemie a vlivu protiepidemických opatření, později též očkování. René Levínský se podílel na přípravě protiepidemického systému PES a byl jedním z členů Mezioborové skupiny pro epidemické situace (MESES). Hlas BISOP, propagující vědecký přístup, byl v médiích nepřeslechnutelný. Jako nezanedbatelný přínos skupiny lze vidět i její multidisciplinaritu, tedy to,  že její výstupy vznikaly jako výsledek diskuse představitelek a představitelů různých oborů.

Názory členů BISOP zněly i v tehdejších zjitřených diskusích, ať už s jednotlivci či skupinami zpochybňujícími hlavní proud vědeckého poznání nebo se zastánci promoření -- BISOP v době, kdy nebylo dostupné očkování, důsledně zastával strategii omezování množství viru v populaci. Působení BISOP se však poněkud překvapivě stalo terčem kritiky i ze strany části vědecké obce, zejména kvůli údajné neprověřenosti používaných modelů a postupů. 

Co se týče vědeckého přístupu a otázky promoření, dává čas členům BISOP postupně za pravdu: různá „alternativní fakta“, jako například tvrzení o zázračnosti ivermektinu nebo o škodlivosti roušek jsou postupně vyvracena a také se ukazuje, že zavádění opatření až v okamžiku zahlcení zdravotnictví vedlo ke zbytečným ekonomickým a lidským ztrátám. Kritiku „rychlé vědy“ je však potřeba nebrat na lehkou váhu a v konkrétních situacích vždy zvažovat, zda čekat na nezpochybnitelné ověření nebo zda jednat. V situaci roku 2020, kdy měla vláda tendenci naslouchat výše zmíněným „alternativám“, byla druhá možnost evidentně tou správnou volbou.

Kniha, kterou držíte v ruce, je sborníkem příspěvků z konference Poučení z pandemie COVID-19 konané v polovině června 2021,\footnote{\url{https://www.bisop.eu/konference-pouceni-z-pandemie-covid-19/}} která si kladla za cíl shrnout aktivity Centra pro modelování biologických a společenských procesů po dobu pandemie. Podobně jako se BISOP celou dobu snažil pokrýt co nejvíce aspektů a nahlížet na ně z různých pohledů, tak se i tento text dotýká mnoha témat a zpracovává je mnoha různými způsoby.

{\em Kapitola 1. Co dnes víme o nemoci covid-19} představuje hlavní postavu dramatu -- nemoc covid-19 -- z pohledu lékařské vědy.
{\em Kapitola 2. Fungují opatření? Korelace versus kauzalita} je reflexí výše zmíněné „války vědeckých faktů“: připomíná, jak obtížné je dokázat příčinnost (kauzalitu), a popisuje, jak se k tomuto problému staví hlavní vědecký proud. {\em Kapitola 3. Senátní volby jako přirozený experiment} se zabývá otázkou, do jaké míry mohl fakt voleb zvýšit počet případů covid-19, přičemž využívá fakt, že se volby konají jen v náhodně vybrané třetině okrsků, což vytváří pro zkoumání unikátní podmínky, které jsou jinak dosažitelné jen při kontrolovaných experimentech.

{\em Kapitola 4. Matematické modelování epidemií: historie a současnost} popisuje základní principy, na nichž stojí většina epidemiologických modelů.
{\em kapitola 5. Epidemiologické modely s agenty} pak pojednává o modelech, které se z těchto principů do jisté míry vymykají, neboť namísto souhrnných počtů pracují s nakaženými jako s jednotlivci. {\em Kapitola 6. Automatická tvorba grafu kontaktů}, {\em Kapitola 7. Simulace epidemiologických opatření v multiagentním modelu} a {\em Kapitola 8. Simulace opatření ve školním prostředí} pak popisují konstrukci a využití agentního modelu, vyvíjeného týmem BISOP.

Další tři kapitoly se věnují očkování. {\em Kapitola 9. Vakcinace pohledem matematických modelů} ukazuje, jak lze očkování pojmout v rámci matematických modelů a co o něm tyto modely mohou říct, {\em kapitola 10. Logistika očkování a prioritní skupiny} popisuje očkovací kampaň v České republice a snaží se odpovědět na otázku, zda byla prováděna optimálně, {\em kapitola 11. Dosáhneme kolektivní imunity očkováním?} se snaží odhadnout, jaké procento populace bude muset být očkováno, aby se země vyhnula podzimní epidemické vlně.

Další kapitoly se věnují konkrétním aspektům dění okolo pandemie. 
{\em Kapitola 12. Systém trasování v ČR: efektivita, reporting a (ne)využívání dat} reflektuje použití jedné z nejúčinnějších zbraní proti pandemii: sledování rizikových kontaktů nakažených, {\em kapitola 13. Compliance jako klíčový parametr průběhu pandemie} se zabývá ochotou občanů dodržovat protiepidemická opatření, {\em kapitola 14. Život během pandemie: Chování, kontakty a jejich proměny} nahlíží pandemii ze sociologického hlediska, {\em kapitola 15. Ekonomické aspekty pandemie} hodnotí dopady pandemie na hospodářství České republiky.

Poslední {\em kapitola 16: Současné výzvy vyžadují reformu způsobu vládnutí} se zamýšlí nad problémy v řízení státu, které situace okolo pandemie odhalila, a naznačuje jejich možná budoucí řešení.





Při čtení příspěvků je třeba mít na paměti, že vznikaly v polovině roku 2021, odrážejí tedy tehdejší stav věcí a tehdejší stav poznání. Ve svém souhrnu však přesto mohou poskytnout ucelený obrázek o prvním roce pandemie, jejím (částečném) poznávání a jejím (ne)zvládání ze strany společnosti. V neposlední řadě jsou dokumentem o působení jedné vědecké iniciativy a do jisté míry dělají jakousi symbolickou tečku za jejím zakladatelským obdobím.


