{\em Kapitola 1. Co dnes víme o~nemoci covid-19} představuje hlavní postavu dramatu -- nemoc covid-19 -- z~pohledu lékařské vědy.
{\em Kapitola 2. Fungují opatření? Korelace versus kauzalita} je reflexí výše zmíněné „války vědeckých faktů“: připomíná, jak obtížné je dokázat příčinnost (kauzalitu), a popisuje, jak se k~tomuto problému staví hlavní vědecký proud. {\em Kapitola 3. Senátní volby jako přirozený experiment} se zabývá otázkou, do jaké míry mohl fakt voleb zvýšit počet případů covid-19, přičemž využívá fakt, že se volby konají jen v~náhodně vybrané třetině okrsků, což vytváří pro zkoumání unikátní podmínky, které jsou jinak dosažitelné jen při kontrolovaných experimentech.

{\em Kapitola 4. Matematické modelování epidemií: historie a současnost} popisuje základní principy, na nichž stojí většina epidemiologických modelů.
{\em kapitola 5. Epidemiologické modely s~agenty} pak pojednává o~modelech, které se z~těchto principů do jisté míry vymykají, neboť namísto souhrnných počtů pracují s~nakaženými jako s~jednotlivci. {\em Kapitola 6. Automatická tvorba grafu kontaktů}, {\em Kapitola 7. Simulace epidemiologických opatření v~multiagentním modelu} a {\em Kapitola 8. Simulace opatření ve školním prostředí} pak popisují konstrukci a využití agentního modelu, vyvíjeného týmem BISOP.

Další tři kapitoly se věnují očkování. {\em Kapitola 9. Vakcinace pohledem matematických modelů} ukazuje, jak lze očkování pojmout v~rámci matematických modelů a co o~něm tyto modely mohou říct, {\em kapitola 10. Logistika očkování a prioritní skupiny} popisuje očkovací kampaň v~České republice a snaží se odpovědět na otázku, zda byla prováděna optimálně, {\em kapitola 11. Dosáhneme kolektivní imunity očkováním?} se snaží odhadnout, jaké procento populace bude muset být očkováno, aby se země vyhnula podzimní epidemické vlně.

Následující kapitoly se věnují konkrétním aspektům dění okolo pandemie. 
{\em Kapitola 12. Systém trasování v~ČR: efektivita, reporting a (ne)využívání dat} reflektuje použití jedné z~nejúčinnějších zbraní proti pandemii: sledování rizikových kontaktů na\-ka\-že\-ných, {\em kapitola 13. Compliance jako klíčový parametr průběhu pandemie} se zabývá ochotou občanů dodržovat protiepidemická opatření, {\em kapitola 14. Život během pandemie: Chování, kontakty a jejich proměny} nahlíží pandemii ze sociologického hlediska, {\em kapitola 15. Ekonomické aspekty pandemie} hodnotí dopady pandemie na hospodářství České republiky.

Poslední {\em kapitola 16: Současné výzvy vyžadují reformu způsobu vládnutí} se zamýš\-lí nad problémy v~řízení státu, které situace okolo pandemie odhalila, a naznačuje jejich možná budoucí řešení.
