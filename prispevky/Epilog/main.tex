%\addcontentsline{toc}{chapter}{Epilog}
\chapter{Epilog}

\textit{Martin Šmíd, BISOP, leden 2022}

\vspace{15mm}

\noindent Hlavní motivací vzniku iniciativ, jako byl Model AntiCOVID-19 pro ČR (později BISOP) či dalších podobných byla snaha o~něco, čemu se anglicky říká evidence-based policy, česky politika založená na faktech, přičemž zmíněná fakta (evidence) mají pocházet z~uznávaného vědeckého výzkumu. Jinak řečeno, politická rozhodnutí by měla mít oporu ve vědě, nikoli jen v~ideologii či okamžitých zájmech. 

K~lednu 2022 byla Česká republika třetí nejhorší z~evropských zemí, a to jak co se týče počtu případů na 100 tisíc obyvatel,\footnote{\url{https://www.statista.com/statistics/1110187/coronavirus-incidence-europe-by-country/}} tak co se týče počtu úmrtí spojených s~covidem,\footnote{\url{https://www.statista.com/statistics/1111779/coronavirus-death-rate-europe-by-country/}} v~první třech měsících roku 2021 byla postupně na třetím, druhém a prvním místě v~počtu nadúmrtí.\footnote{\url{https://ec.europa.eu/eurostat/databrowser/view/DEMO_MEXRT__custom_309801/bookmark/table?lang=en&bookmarkId=22df2744-9f37-4f0e-831f-bfe32824397d}} V~tomto kontextu není nijak nadsazené mluvit o~selhání a přirozeně vyvstává otázka po jeho příčinách. 


Jako možný důvod tohoto stavu se přirozeně nabízí absence politiky založené na faktech. Věc je však složitější. Nedá se totiž říci, že by se tehdejší vláda Andreje Babiše, známá svou proklamovanou neideologičností a pragmatismem, tomuto principu bránila. V~počátcích pandemie se díky jednoduchému matematickému modelu Pavla Řeháka premiér nechal přesvědčit k~zavedení opatření, která vedla k~účinnému potlačení první vlny epidemie. S~přibývající nespokojeností obyvatel se zavedenými opatřeními sice ochota vlády k~ráznějším opatřením oslabovala, ale i to bylo často ospravedlňováno názory expertů, jakkoli šlo o~\uv{disentní} názory, které se později ukázaly jako problematické. A~vždy, když se situace začala zhoršovat a hrozilo zahlcení zdravotnictví, vláda se znovu obrátila na \uv{mainstreamové} vědce.

Příčiny našeho selhání jsou hlubší. Těžko je všechny vystopovat, svou roli však zcela jistě sehrál \uv{smrtící koktejl}, sestávající z~nevyřčené vůle k~rychlému promoření, absence široce respektovaných autorit a neschopnosti státu řešit komplexní situace.

\section*{Promoření versus omezení}

Jedním z~největších a těžko řešitelných problémů, před kterými společnost v~pandemii stojí, je rozdělení zátěže, kterou pandemie přináší, mezi jednotlivé skupiny obyvatelstva. Podobně jako u~přerozdělení bohatství zde nejde o~vědecký problém, který by měl optimální řešení, nýbrž o~vyvažování protichůdných zájmů (tradeoff). Ochránit zdraví ohrožených skupin lze totální uzávěrou, minimalizovat ekonomické škody lze za cenu obětování ohrožených skupin. Toto jsou ale extrémní řešení, je třeba volit nějaký kompromis. Věda může vysvětlit, jaké kompromisy jsou možné a které z~nich jsou v~každém ohledu horší než jiné, konečná volba je však věcí politického rozhodnutí. 

Mnohé nasvědčuje tomu, že si česká společnost, anebo přinejmenším její část, poté, co se možnost úplného vymýcení viru ukázala jako iluzorní, vybrala cestu rychlého promoření s~vidinou co nejrychlejšího návratu do normálu, byť za cenu značného počtu obětí na zdraví a životech a následně i ekonomice. Chování vlády, která vždy „tahala za brzdu“ až pod hrozbou kolapsu zdravotnictví, tuto domněnku nepřímo dokazuje. Pro spekulaci, že šlo o~tichý společenský konsenzus, svědčí i fakt, že opačný názor, totiž že je třeba virus když ne vymýtit, tak alespoň omezit, si netroufla propagovat žádná z~politických stran, kandidujících ve volbách v~říjnu 2021. 

To, že promořování k~proklamovanému návratu do normálu nevede, je ve světle nástupu varianty omikron, která dokáže i postinfekční imunitu celkem lehce obejít, více než jasné. Co k~němu naopak vést může, je koncept takzvané hybridní imunity, spočívající v~co největším proočkování a následném \uv{puštění viru do populace} -- očkování totiž výrazně snižuje riziko těžkých průběhů, a to pro všechny dosud známé varianty. V~tomto světle se státy, které až do doby, kdy bude očkovanost dostatečná, infekci omezovaly, vyhnuly mnoha zbytečným obětem na životech a zdraví, zatímco ty, které promořovaly, nic nezískaly, naopak spíše ztratily,  protože epidemii pod hrozbou kolapsu zdravotnictví nakonec stejně musely zabrzdit, a to za podstatně vyšších nákladů, než kdyby to byly udělaly včas.

Hlavní proud epidemiologie, u~nás kromě centra BISOP reprezentovaný sdružením MeSES či Iniciativou Sníh, během prvního roku pandemie poměrně jednoznačně prosazoval cestu omezování viru, byť byly jeho hlavní argumenty spíše etické (posláním lékaře je léčit) či lékařské (poukazující na zdravotní rizika nemoci), přičemž panovalo optimistické očekávání, že pandemii zastaví dostatečná proočkovanost. I~když se ve světle zjištění, že účinnost současných vakcín v~čase poměrně rychle klesá, koncept omezování viru  poněkud zrelativizoval, zpětně se i tak jeví jako správný, protože většímu počtu lidí umožnil dožít se možnosti ochrany hybridní imunitou. 

Je dobré si všimnout faktu, že strategii promořování (alespoň v~naší zemi) nikdo z~oficiálních představitelů veřejně nepropagoval, snad jen s~výjimkou tehdejšího náměstka ministra zdravotnictví Romana Prymuly, což ovšem premiér Babiš odmítl s~odůvodněním, že je to \uv{příliš riskantní}.\footnote{\url{https://www.novinky.cz/domaci/clanek/babis-se-postavil-za-hamacka-nesouhlasi-s-prymulou-ohledne-promoreni-40319551}} Promořování otevřeně nepodporovali ani \uv{vědečtí disidenti}, tedy oponenti hlavního proudu z~řad vědecké obce. Ti kteří se omezili na všemožnou kritiku oficiálního postupu, ať už šlo o~přeceňování stavu kolektivní imunity,\footnote{\url{https://www.idnes.cz/zpravy/domaci/koronavirus-covid-vojtech-thon-testovani-studie-promorenost.A201027_105917_veda_vov}} přeceňování možnosti léčení už propuklých případů,\footnote{\url{https://cnn.iprima.cz/expert-varuje-pred-plosnym-ockovanim-nesmi-byt-soubeh-infekce-a-vakciny-51177}} včetně propagace „zázračných“ léků jako Isoprinosin\footnote{\url{https://www.zdravotnickydenik.cz/2021/01/isoprinosine-zrejme-neni-zazracny-lek-proti-covid-19-jeho-ceskou-studii-provazi-vazna-eticka-pochybeni/}} či Ivermectin,\footnote{\url{https://vtm.zive.cz/clanky/tecka-ivermectin-pred-covidem-neochrani-nekolik-vedeckych-studii-ktere-to-tvrdily-bylo-stazeno/sc-870-a-213213/default.aspx}} zpochybňování účinnosti protiepidemických opatření\footnote{\url{https://smis-lab.cz/wp-content/uploads/2021/06/2021-06-07-Straka.pdf}} či spekulace o~skrytých nežádoucích účincích vakcín.\footnote{\url{https://www.parlamentnilisty.cz/arena/rozhovory/Tajena-data-1-umrti-po-vakcinach-hlaseno-Cesky-matematik-citi-zmar-685018}} Tyto \uv{kritické} hlasy, které se často nevyhýbají domněnkám, ve valné většině nemají potřebnou vědeckou úroveň, nejsou nijak oponovány, přitom nenabídly ke stávajícím opatřením žádnou konzistentní alternativu, snad jen s~výjimkou vágních návrhů na neomezování zdravých a ochranu zranitelných\footnote{\url{https://plus.rozhlas.cz/ze-se-covid-siri-v-populaci-asi-nikomu-nevadi-musime-chranit-hlavne-seniory-8349293}}, což však, jak se ukázalo například ve Švédsku, v~praxi může fungovat jen těžko. 

Společnost se tak ocitla ve schizofrenní situaci, kdy její vláda sice navenek mluvila o~boji s~pandemií a ochraně zdraví, ve skutečnosti však proti ní dělala jen nezbytné minimum, patrně aby tak vyšla vstříc předpokládané touze společnosti po rychlém promoření. To, že by společnost dobře obhájenou a konzistentní politiku marginalizace viru nepřijala, však vůbec není jisté.

\section*{Absence autorit a nedůvěra}

Jak píšeme již v~kapitole \ref{Veda_pandemie}, čekat od vědy definitivní odpovědi na všechny otázky je naivní, a to zvláště v~situacích, kdy se otázky týkají nějakého nového fenoménu. To ovšem neznamená, že neexistují vědci, kteří \uv{co řeknou, to platí}, byť za cenu občasného přiznání, že odpověď neznají. 

Zatímco Německo má svůj Kochův institut a Christiana Drostena, Spojené státy svoje CDC a Anthonyho Fauciho, u~nás žádná výrazná institucionální ani personální autorita, na kterou by bylo lze se ve věcech epidemie spolehnout, neexistuje. Nejblíže k~pozici hlavní autority ve věci covidu měl jistě dostatečně erudovaný Roman Prymula, ten si však kromě zmíněné epizody s~promořováním svou pozici podkopal tím, že nedokázal dodržovat opatření, která sám prosazoval.\footnote{\url{https://www.irozhlas.cz/zpravy-domov/roman-prymula-fotbal-slavie-leicester-covid-19-lockdown_2102182139_kro}} Kandidáty na tuto neoficiální pozici by teoreticky mohli být Petr Smejkal a MeSES, ti však přišli příliš pozdě, tj. v~situaci, kdy už byla společnost názorově rozdělena.

To, že se po celou dobu pandemie k~věci vyjadřovala řada expertů s~často diametrálně odlišnými názory, nutně muselo uvést laiky ve zmatek. V~zemích s~centrální epidemiologickou autoritou sice kontroverzní názory také hojně zaznívají, lidé však alespoň vědí, co říká hlavní proud vědy -- naše veřejná diskuse tuto „kotvu“ postrádá. Situace je o~to horší, že se do odborné diskuse zapojují i lidé, kteří sice mají nezpochybnitelný odborný kredit, ale v~jiném oboru, takže není vždy na první pohled zřejmé, že se v~epidemiologii nevyznají. Dalším problémem je, že média často dají kontroverznímu názoru prostor jen proto, že je alternativou jiného názoru, a přitom málo zkoumají jeho validitu či důvěryhodnost jeho autora. Laik, zejména pokud je zvyklý autoritám věřit, se tak nutně musí cítit ztraceně a dá se předpokládat, že se přikloní na tu stranu, která mu více vyhovuje, nehledě na pravdivost.

Zmíněný zmatek je přiživován dezinformačními aktivitami, zejména šířením fa\-leš\-ných zpráv (fake news) a různých konspiračních teorií, a to prostřednictvím sociálních sítí,\footnote{\url{https://www.investigace.cz/zpravy-roku-2021/}} řetězových e-mailů a dezinformačních serverů, kteřé často citují výše uvedetné \uv{disentní} vědecké názory. \footnote{\url{https://denikn.cz/785894/vedci-proti-vedcum-proc-se-cast-expertu-plavi-na-jedne-lodi-s-dezinformatory/}}

Dalším faktorem, který podkopal důvěru v~jakékoli \uv{oficiální} autority ve věci covidu, byla chaotická politika minulé vlády, která po úspěchu z~jara 2020 evidentně podlehla iluzi, že je pandemie věcí minulosti, a navzdory mnoha varovným hlasům se téměř nijak nepřipravila na podzimní vlnu nákaz, jejíž možnost v~podstatě nepřipouštěla. Ač je to neuvěřitelné, stejné selhání zopakovala další rok včetně premiérových slov o~údajném zvládnutí pandemie,\footnote{\url{https://twitter.com/andrejbabis/status/1432951310911950850?lang=cs}} ačkoli mnozí včetně skupin BISOP a MeSES varovali, že je při tehdejší malé míře očkovanosti podzimní vlna v~podstatě nevyhnutelná. Za očkování se vláda sice postavila, ale prostřednictvím své propagační kampaně ho prezentovala jako definitivní tečku za pandemií, ačkoli bylo od začátku jasné, že imunita získaná očkováním se může v~čase snižovat. 

Zmíněné skutečnosti podkopaly důvěru obyvatel k~jakýmkoli protiepidemickým opatřením, a to v~čase, kdy je jasné, že pandemie zdaleka není u~konce, byť snad má svou nejhorší fázi za sebou.

\section*{Tisíc malých věcí}

Jednu hlavní příčinu toho, že jsme v~rámci Evropy jedněmi z~\uv{worst in
covid}, najdeme těžko. Spíše se dá říct, že jde o~\uv{tisíc malých věcí}.

Namátkou: V~Británii je v~lékárnách pro každého k~dispozici prakticky neomezené množství antigenních samotestů, ze kterých se okamžitě pozná výsledek, zatímco u~nás je v~rámci prevence hrazeno pět PCR testů, které jsou sice přesnější, ale výsledek se testovaný dozví až druhý den, přičemž neočkovaným se žádné testy nehradí. 

Ve Skotsku jsou již od začátku občané zváni na očkování formou opt-out, to jest dostanou pozvánku k~očkování s~termínem, na který nemusí přijít, zatímco u~nás byli i senioři nuceni registrovat se pomocí online systému a o~opt-outu se začíná teprve diskutovat. Přitom právě tento drobný detail může mnoho nerozhodných k~očkování motivovat.

Samostatnou kapitolou je konzistence, komunikace a vymáhání protiepidemických opatření. V~žádné z~těchto oblastí si Česká republika nevede dobře. Úřady za celou dobu pandemie nedokázaly vytvořit webový server, kde by bylo jasně a srozumitelně sděleno, jaká opatření v~danou chvíli platí, místo toho jsou informace rozesety po serverech jednotlivých resortů, mluví nesrozumitelnou úřední řečí a obsahují nepřehledné množství výjimek. Nekonzistence opatření se možná dala omluvit v~hektických počátcích pandemie, ale například to, že na podzim 2021 byly zakázány venkovní vánoční trhy, zatímco provoz ve vnitřních prostorách nákupních center nebyl nijak omezen a mohly se konat trhy farmářské, je na pováženou. Nedodržování a obcházení pravidel je běžnou praxí, což potvrzují jak anekdotické evidence, tak data z~výzkumů veřejného mínění.\footnote{\url{https://www.paqresearch.cz/post/postoje-k-opatrenim-a-zkusenosti-s-postupem-zamestnavatelu}}

Očkování i testování by mělo být maximálně geograficky dostupné jako v~Ně\-mec\-ku. Oficiální zdravotnické autority by měly mluvit srozumitelně a navzájem konzistentně jako v~Izraeli, pomohou i maličkosti, jako když má revizor či policista u~sebe roušky pro ty, kdo si je zapomněli doma. Podobných příkladů by bylo možné najít mnoho.

Všechny tyto „malé věci“ mají jeden společný rys a současně důvod, proč se je nedaří zavádět: ve srovnání s~plošnými opatřeními jsou organizačně náročnější a vyžadují určitou flexibilitu státu. Zastavit první vlnu epidemie pomocí uzávěry jsme dokázali, projekt propojeného systému zahrnujícího testování, trasování a elektronické sledování rizikových kontaktů, nazývaný chytrá karanténa, se však zadrhl, zejména na neochotě zainteresovaných složek spolupracovat. Plošné testování ve firmách dlouho ztroskotávalo na faktu, že ač by se celospolečensky vyplatilo, příslušné resorty jej odmítaly, protože by znamenal výdaje z~jejich omezeného roz\-poč\-tu. Podobné důvody byly i za dlouhou neochotou k~navýšení nemocenské za pobyt v~karanténě a izolaci tak, aby se jim lidé z~ekonomických důvodů nevyhýbali. Jindy byla překážkou nepružná legislativa, jako příklad uveďme mizivou proočkovanost dětí do 11 let, jedním z~jejíchž důvodů je předpis, že tyto děti mohou být očkovány pouze pediatry. 

\section*{Závěr}

To, že se některé tyto malé, ale často i větší věci podařilo prosadit díky úsilí mnoha altruistických dobrovolníků, není třeba připomínat. Vyvstává však otázka, jaký vliv měla práce skupin snažících se poskytovat vstupy pro \uv{evidence based policy}, jako je například BISOP.  Tuto otázku lze těžko zodpovědět, protože kromě oficiální spolupráce s~ministerstvem školství týkající se znovuotevírání škol na jaře 2021 se výstupy BISOP omezily na odborné články, konferenční příspěvky, situační zprávy a početné mediální vstupy jednotlivých členů. Nedá se spolehlivě měřit, co z toho a jakou měrou k~řešení problémů spojených s~pandemií přispělo. Zbývá jen věřit, že snad ano.


