\chapter{V\v{e}da za \v{c}as\r{u} pandemie}
\label{Veda_pandemie}

\textit{Josefína Weinerová, Josef Šlerka}

\vspace{15mm}


\noindent Pandemie SARS-CoV-2 s~sebou přinesla řadu výzev pro vědu a vědeckou komunikaci. S~tím, jak přibývaly počty pozitivních případů v~různých zemích, vzrostl také zájem o~výsledky vědeckých studií. Veřejnost tak měla možnost v~reálném čase sledovat nejen proces vědeckého bádání, ale také to, jak jeho závěry ovlivňují každodenní život skrze zavedená opatření.

Pozornost veřejnosti trvá již dva roky. Kromě soustředěného zájmu médií o~pokroky v~oblasti vědy má pandemie také vliv na fungování samotných médií. Koronavirová krize stvořila nové mediální autority v~oboru a vytvořila úctyhodný archiv jejich vyjádření. Tento příliv informací ukázal, že vědecké závěry, které se v~médiích objevují, si mnohdy protiřečí. V~rámci této kapitoly se zamýšlíme nad tím, jak pandemie ovlivnila vědu a její medializaci a jaké existují dichotomie mezi jazykem vědy a médií. Našim cílem není podat vyčerpávající analýzu, nýbrž ukázat oblasti, které pandemie dobře nasvítila.

\subsection*{Věda a pandemie}

Od počátku pandemie nabrala věda bezprecedentní rychlost. Například počet zaslaných medicínsky zaměřených vědeckých článků do impaktovaných časopisů Elsevier stoupl mezi únorem a květnem 2020 o~63 \% oproti stejnému období v~roce 2019 \cite{Squazzoni2021}. K~9. lednu 2022 mělo zadání slova \uv{COVID-19} do vyhledávače \url{PubMed.gov}\footnote{PubMed je databáze citací biomedicínských článků spravovaná Národní knihovnou medicíny Spojených států amerických: \url{https://pubmed.ncbi.nlm.nih.gov/about/}} 215 283 výsledků (pro porovnání, zadání hesla \uv{EBOLA} má \uv{jen} 10 723 výsledků).
Toto publikační úsilí se rozhodně vyplatilo. Covid-19 se stal jednou z~nejlépe prostudovaných nemocí všech dob, a to pro vědu v~neuvěřitelně krátké době dvou let. Diagnostické testy jsou schopné detekovat přítomnost viru v~rámci minut, máme k~dispozici ne jednu, ale několik funkčních a schválených vakcín (a další jsou v~procesu schvalování), masivní databáze sekvenačních dat z~celého světa nám umožňuje do dosud nemyslitelného detailu sledovat evoluci nového viru \cite{Maxmen2021} a v~neposlední řadě pandemie podnítila vznik řady mezioborových skupin, do kterých se řadí i Centrum pro modelování biologických a společenských procesů (BISOP).

Pandemie však také odhalila slabiny vědy, které doposud mohly být většinou vědecké obce a veřejnosti přehlíženy. RetractionWatch, iniciativa monitorující články, které byly staženy z~vědeckých publikací například z~důvodu manipulace s~daty, k~datu 9. ledna 2022 evidovala 206 studií o~onemocnění covid-19 \cite{RetractionWatch}. A~to jsou jen studie, u~kterých se na manipulaci s~daty opravdu prokazatelně přišlo. Ačkoli je tento počet alarmující, je nutné podotknout, že proces retrakce článků často trvá roky. To, že se na tyto články přišlo tak rychle a byly staženy z~oběhu, je svým způsobem pozitivní a ukazuje to míru pozornosti, kterou vědecká obec studiím covidu-19 věnovala a věnuje.

V~průběhu pandemie se také několikrát stalo, že se v~médiích postupně objevovaly protichůdné vědecké závěry. Příkladem je třeba lék Ivermectin, který byl nejdříve zobrazován jako možná naděje pro léčbu covidu \cite{Ceskatelevize2020}, ačkoliv později větší studie tyto závěry nepotvrdily \cite{Lopez-Medina2021}. K~pochopení toho, proč jsou navzájem si protiřečící studie pro vědu normální a přínosné, je třeba se zaměřit na principy, které tvoří tzv. vědeckou metodu a které z~vědy udělaly zatím nejúspěšnější systém pro popis reálného světa.

Vědecká metoda, tak jak se používá dnes, se částečně odvíjí od filosofie falzifikace, kterou zformuloval Sir Karl Popper \cite{Popper2002}. Od jiných (ne\-vě\-de\-ckých/pseu\-do\-vě\-de\-ckých metod) se odlišuje zejména tím, že se nesnaží hypotézy potvrdit, nýbrž vyvrátit.\footnote{Ve vědecké praxi to znamená, že se u nasbíraných dat zkoumá, zda by mohla pocházet z takové populace, kde postulovaný efekt neexistuje. Pokud je pravděpodobnost, že by data pocházela z~populace, kde efekt neexistuje, opravdu malá, teprve říkáme, že jsme selhali ve vyvrácení nulové hypotézy. Zde se vědecká praxe poněkud odchyluje od Popperovy filosofie, protože v takovém případě je výsledek brán jako podpora pro postulovanou alternativní hypotézu.} Vědec na začátku předpokládá, že naměřený jev ve skutečnosti neexistuje. Počítá, jak velká je pravděpodobnost, že tento jev v~datech vznikl náhodou. Pokud je tato pravděpodobnost za určitou předem určenou hranicí, říkáme, že jsme zamítli nulovou hypotézu (hypotézu, že daný jev neexistuje). I~tak ale nadále existuje určitá pravděpodobnost, že jsme se dopustili chyby I. druhu a zamítli jsme nulovou hypotézu, ačkoli byla pravdivá. Pravděpodobnost této chyby se zmenšuje s~tím, jak se zvyšuje počet studií, které detekují stejný jev. Proto to se, jestli daný jev skutečně existuje (a nulovou hypotézu jsme tudíž zamítli oprávněně), dá s~dostatečnou jistotou určit až na bázi výsledků z~řady různých studií stejného tématu.

Z~toho vyplývají dvě charakteristiky vědy: a) je to proces veskrze pomalý a b)~omyl ve vědě (tedy to, že se určitý výsledek v~pozdějších studiích nepotvrdí) je něco, co je pro vědu přirozené a hlavně cenné. Výsledky jednotlivých studií jsou tak pro vědu zajímavé ve většině případů pouze v~kontextu dalších studií na stejné téma.

To, že se stav poznání mění v~čase, také znamená, že vědecká vyjádření se musí interpretovat s~ohledem na to, kdy byla vyřčena. Pokud se u~nějakého léku vědci na základě dostupných dat shodnou, že není dostatek důkazů o~jeho účinnosti, to neznamená, že je to stále pravdivý výrok o~půl roku později. S~přibývajícími daty se během posledního roku měnila míra poznání například o~tom, které demografické skupiny je bezpečné naočkovat, nebo o tom, do jaké míry očkování funguje pro různé varianty viru. Příklady lze najít i v~historii. Charles Darwin nemohl ve své teorii evoluce popsat mechanismus genetické dědičnosti z~toho prostého důvodu, že existence genů nebyla v~té době ještě široce známá.\footnote{Charles Darwin sumarizoval teorii evoluce ve své knize O původu druhů publikované v roce 1859. Gregor Mendel na svých pokusech s hrachem v té době již pracoval, nicméně výsledky publikoval v roce 1866.}

Kromě popperovské tradice falzifikace se věda opírá ještě o~koncept vědecké komunity. Doby, kdy jeden vědec byl schopen pojmout veškeré vědění své éry, jsou dávno pryč. Vědecké poznání se dnes sestává de facto ze sítě poznatků, jejichž držitelem je vědecká komunita. Tato tzv. networked knowledge stojí v~jádru exploze vědeckého poznání (viz např. Jonathana Raucha a jeho knihu The Constitution of Knowledge: A~Defense of Truth). Důraz na práci vědecké komunity ještě více posiluje její popperovské základy. Síť poznatků totiž potřebuje nutně dva předpoklady, které z~ní vyplývají. Nikdo nemá poslední slovo -- každé tvrzení platí jen do svého vyvrácení, a nikdo nemá osobní autoritu, kterou by definitivně \uv{posvětil} pravdu. Rauch doslova říká: \uv{Můžete tvrdit, že nějaké tvrzení bylo prokázáno jako znalost, pouze do té míry, do jaké metoda použitá k~jeho kontrole dává stejný výsledek bez ohledu na identitu kontrolujícího a bez ohledu na zdroj tvrzení. Ať už uděláte cokoli, abyste výrok ověřili, musí to být něco, co může alespoň v~principu udělat kdokoli a získat stejný výsledek. Na nikoho, kdo navrhuje hypotézu, se neberou zvláštní ohledy jen proto, kdo je nebo k~jaké skupině patří. Na tom, kdo jste, nezáleží; pravidla platí pro všechny a osoby jsou zaměnitelné. Pokud vaše metoda platí pouze pro vás nebo vaši spřízněnou skupinu či lidi, kteří věří stejně jako vy, pak se nezakládáte na realitě} \cite{Rauch2021}.

Vědcem je tak dnes člen komunity, která dodržuje určité společné procedury. Lidé v~této komunitě přitom neustále zmenšují svoji osobní autoritu a dávají ostatním možnost jakékoli tvrzení ověřit či vyvrátit.

K~tomu je třeba doplnit ještě jednu změnu v~tom, jak provozujeme vědu. Netýká se ani tak metod vědy, jako prostředí, ve kterém je věda usazena. Tou změnou je přechod k~intenzivnímu shromažďování dat na jedné straně a možností, které přináší jejich počítačové zpracování, na straně druhé má současná věda v~řadě případů k~dispozici bezprecedentně velký objem dat o~zkoumaných fenoménech. Například v~případě pandemie covid-19 jsou to záznamy o~jejím průběhu na úrovni jednotlivých pacientů. Pokud jde o~jejich zpracování, je tu hned dvojí změna. Tou první je sdílení dat v~celé komunitě vědců. Druhou změnou je pak relativně nedávný nástup metod spojených nejen s~komputací.

Někteří v~tomto kontextu mluví dokonce o~změně celého paradigmatu vědy, jako například autoři knihy The Fourth Paradigm \cite{Hey2009}. Ti popisují pohyb vědy od empirické, založené na pozorování přírody. Na ni navazuje věda teoretická, která se snaží vytvořit model fungování přírody.  Tu ve dvacátém století doplňuje přístup komputační a následně datový. Tyto přístupy jsou umožněny právě rozvojem počítačů a dostatkem dat. Specifické postavení v~této změně získávají komputační modely.

Modelování představuje poznávací přístup, který je založen na nepřímém poz\-ná\-vá\-ní a jako metoda doplňuje experimenty. V~rámci některých otázek týkajících se pandemie je možnost experimentování velmi omezená. V~první řadě je zde důvod etický, není možné například izolovat města plná nakažených a pozorovat, jak se mění parametry pandemie podle míry omezení sociálních kontaktů. Do hry tak vstoupil přístup, který realitu simuluje tak, aby na ní mohl dělat experimenty. Modelování umožňuje nejen odhad parametrů, které pohánějí pandemii v~současnosti, ale zejména také vytváření predikcí budoucího vývoje.

Různé fenomény v~rámci pandemie tak vyžadují použití různých vědeckých pří\-stu\-pů. Data o~šíření viru v~různých zemích nám umožňují těžit z~rozvoje počítačového zpracování dat
a vytvářet možné predikce pro konkrétní podmínky. Pandemie je primárně společenský fenomén. Pro společnost samotnou je v~tomto ohledu důležité nejen vědecké poznání, ale hlavně předpovědi možného vývoje, které se mohou stát základem pro rozhodování politiků i občanů.

\subsection*{Věda a politika}

Věda sama nemá prostředky, aby rozhodovala o~tom, co se má dělat. Věda umí předkládat možnosti, nikoli rozhodnutí. Jinými slovy: věda je deskriptivní a prediktivní, nikoli normativní. Umí říct, že snížit dynamiku šíření pandemie je možné tak, že se přijme jedno z~následujících řešení, ale nemůže říci, které je lepší. Není politikou a ani nemůže být.

Politika je přijímání rozhodnutí a s~tím spojené přijímání odpovědnosti za tato rozhodnutí ve volebních cyklech. Pro politické rozhodování je důležité nalezení rovnováhy mezi závazky vůči voličům. Dále pak řešení problémů, které nastávají mimo původní závazky, a očekávaný zisk hlasů v~dalších volbách na základě těchto rozhodnutí a jejich dopadů. Čím blíže je volbám minulým, tím více hrají roli závazky, a čím blíže k~novým, tím více očekávané hlasy. Navíc je politika vázána rozhodnutími, kterými se snaží dosáhnout pokud možno maximálního konsenzu.

V~ideálním světě, kde by byla věda v~dokonalém souladu s~politikou, by rolí vědy bylo předkládat politikům doporučení a možné scénáře dalšího vývoje s~tím, že by politici přijímali svá rozhodnutí na jejich základě a vznikl by tak prostor pro politiku založenou na faktických informacích tzv. evidence-based policy.

Ve skutečnosti se ale stalo něco jiného. Jak již zmiňujeme výše, rychlost výzkumu kolem koronaviru způsobila, že bylo možné v~jednu chvíli citovat řadu studií, jejichž výsledky byly ve vzájemném protikladu. Selektivní použití vědeckého poznání se tak stalo nástrojem legitimizace různých zájmových tlaků a vědci sami nebyli schopni se proti tomu úspěšně bránit. Z~vědy se v~rukou politiků stal nástroj pro potvrzování vlastních stanovisek, nikoli nástroj pro ověřování účinnosti kroků, kterým se věda mohla stát.

Navíc ve chvíli, kdy se vědci začali objevovat v~médiích nejen jako pasivní informátoři o~pandemii, ale jako aktivní komentátoři pandemie, stali se i komentátory politických rozhodnutí z~pozice svých vědeckých oborů. A~to nebyla jediná nová událost z~pohledu médií, kterou přinesla pandemie.

\subsection*{Věda a média}

Nejen pro vědu a politiku byla pandemie trvající dva roky zcela nová situace. Ideální rolí médií v~demokratických systémech je informovat občany o~dění ve světě pokud možno tak, aby se sami mohli dobře rozhodovat. Média mají své žánry (zprávy, komentáře, přímé výstupy z~jednání apod.) a podávání pravidelných a aktuálních informací o~krizích pro ně není novinka. Není tedy překvapivé, že bezprostředně s~nástupem pandemie mediální domy spustily mimořádná vysílání, speciály na we\-bech a k~celé situaci přistupovaly tak, jako v~minulosti například k~referování o~povodních.

Pandemie však bohužel neměla charakter povodní, ale vleklé choroby, v~níž se střídají světlé a stinné okamžiky. Na to novinářská obec naprosto nebyla připravena a celkově to vytvářelo místy zcela neadekvátní reakce. Dlouhodobost pandemie přinesla v~prvé řadě rozsáhlý problém, jak referovat o~datech z~jejího průběhu. Každodenní zpravodajství o~počtu nakažených připomínalo spíše meteorologické zpravodajství. Informovalo o~denních číslech, aniž by je zasazovalo do nějakého kontextu, a teprve postupně se učilo zohledňovat například cykly v~testováních či fakt, že denní data neposkytují dostatečný obraz situace. To však nebyl jediný problém, my se v~našich poznámkách zaměřujeme především na momenty, které se týkají styčných ploch mezi vědou a mediálním světem.

Tržní logika mediálního trhu vede média k~tomu, že potřebují přinášet stále nové informace a pokud možno obměňovat mluvčí, případně i názory, tak aby byla zaručena názorová a obsahová pestrost. To se ovšem může ukázat jako kontraproduktivní při intenzivním a zároveň dlouhodobém problému. Vytváří se tak totiž zcela falešný obraz toho, co o~pandemii víme a jaký panuje okolo ní momentální konsenzus. Přednost tak postupně dostávají názory, které jsou často marginální a~dostatečně jiné, často dokonce přímo nevědecké.

Protože novináři většinou nevědí, jaký je současný stav konsenzu vědecké komunity, dávají prostor jednotlivým názorům spíše na základě svého dojmu. Případně se řídí logikou rovného zastoupení, která velí, aby v~mediálním diskurzu byly zastoupeny všechny možné názory, tak jako je to v~případě politiky. Díky tomu se stává, že ačkoli je nějaký názor zastoupen ve vědecké komunitě třeba 5 \%, tak mediálně může být zastoupení násobně vyšší. Politici, kteří operují na pomezí informací od vědců a mediálního obrazu pandemie, pak musí volit různé strategie, jak v takto vymezeném prostoru vystupovat.

Vzhledem ke značnému počtu odborných článků a jejich snadné dostupnosti také média mnohokrát informovala o~aktuálních výzkumech. Často o~studiích, které teprve v~preprintech čekaly na své přijetí do odborných publikací, novináři psali jako o~jasných závěrech. Když se pak některé výzkumy ukázaly jako neprůkazné, nebyla média schopna o~tom přiměřeně referovat. Hraničním případem může být aktivní propagace užívání ivermektinu na serveru Novinky.cz, která vedla až k~hrozbě pokutou od SÚKL.

Popis a zarámování pandemie v~médiích mají navíc vliv na chování veřejnosti a to pak může dále ovlivnit průběh pandemie. Kupříkladu v~médiích vyřčená predikce vysokého počtu nakažených a hospitalizovaných může působit na chování lidí, což ovlivní míru interakcí a způsobí, že se predikce nevyplní.

Dalším problémem je fakt, že k~vědě nepatří jen závěr, k~němuž vědec dochází, ale také vědecká metoda, která k~němu vedla. Při velkém počtu oslovených expertů a požadované  různosti názorů byl mediální prostor tak zahlcen nabídkou predikcí, že jedna z~nich nejspíše nakonec \uv{správně} předpovídala výsledek, aniž byla tato predikce nutně vědecky podložená.

Na straně vědy byl od počátku problém v~rozdílném jazyce. Vědci byli postaveni před úkoly, které obvykle řeší spíše pomalu a s~komunikovanou mírou nejistoty. V~případě pandemie jsou takovým úkolem například možné scénáře vývoje podle případných parametrů šíření a odhadovaných výsledků. I~když se vědci často snažili mluvit o~možných scénářích, které mají nějakou pravděpodobnost, bylo pro ně velmi složité mluvit o~nich s~novináři, kteří mají v~oblibě spíše jasné odpovědi než ty plné, exaktně definované nejistoty.

Za celou dobu pandemie byl jen málokterý vědec v~médiích konfrontován se svými předpověďmi. Ve vědě je odhalení chyby v~modelu zásadním krokem vpřed. Debata o~tom, proč model nevyšel, vede k~jeho vylepšení. To však bylo a je proti logice médií, zvlášť pokud by média konfrontovala předpovědi, které sama přinášela, vystavovala by se vzniku pocitu, že svým čtenářům lhala a že se stala nějaká fatální chyba. Přitom by taková veřejná konfrontace s~neúspěšnými předpověďmi vedla k~lepšímu pochopení práce vědy jako takové.

Obvykle vědec pracuje s~nějakými prav\-dě\-po\-dob\-nost\-mi a scénáři. Ty umí vyjádřit číselně, ale při komunikaci s~médii bývá často v~pokušení nekomunikovat rozložení pravděpodobnosti celé. Takže i když vidí, že existuje desetiprocentní prav\-dě\-po\-dob\-nost, že budou například nemocnice plné, a desetiprocentní prav\-dě\-po\-dob\-nost, že bude vše v~naprostém pořádku, někteří raději volí katastrofičtější scénáře. K~tomu je třeba přidat fakt, že média dávají přednost spíše těm, kteří říkají svá tvrzení jasně a bez náznaků nejistoty.

Ostatně komunikace nejistoty vyjádřené čísly pomocí slov představuje další z~pro\-blé\-mů, které před sebou vědci komunikující s~médii mají. Kolik procent je slušná šance? Kupříkladu i v~něčem tak zdánlivě etablovaném, jako je referování o~počasí ve zprávách, si lidé často neuvědomují, kolik které označení znamená v~procentech. Například \uv{ojediněle} znamená na 5 \% až 29 \% území \cite{Ceskyhydrometeorologickyustav}.

V~českém prostředí je v~rámci pandemie citelná absence jedné instituce či osoby, která by svou komunikací zastupovala centrální vědecký proud a zastávala tak roli hlavního zdroje důvěryhodných informací. Již dříve zmíněná snaha médií dát stejný prostor všem (i minoritním) názorům tak může vytvářet dojem, že všechny prezentované názory jsou i stejně důvěryhodné. Tomu samozřejmě neprospívá ani fakt, že za dobu pandemie ministerstvo zdravotnictví, které by teoreticky takovou autoritou mohlo být, již pětkrát změnilo svého ministra (k~datu 9. ledna 2022).

\subsection*{Závěr}

Věda od počátku pandemie zažila nezpochybnitelné období triumfů: od rychlosti, s~jakou byl dekódován a publikován genom viru SARS-CoV-2, přes rychlost vývoje a testování vakcín až po vznik nových interdisciplinárních vědeckých spoluprací. Věda v~průběhu pandemie bezpochyby prokázala svou užitečnost. Na druhou stranu se ale během pandemie projevily i mnohé slabiny vědy z~pohledu jejích procedur v~případě komunikace s~veřejností a zejména při interpretaci jejích závěrů.

Základem je práce vědy s~hypotézami a existence vědecké komunity, hledající konsenzus. Tedy neustálé přehodnocování již objeveného. Vědci nejsou neomylní, ale zásadní rozdíl mezi vědou a jinými druhy lidského poznání je, že věda se svou chybovostí předem počítá, aktivně své omyly hledá a v~případě, že omyl najde, své závěry mění. Pro širokou veřejnost ale neshoda v~rámci vědeckých závěrů spíše podrývá důvěru ve vědu jako celek.

V~období pandemie bylo také možné sledovat, jak se vědecké závěry a predikce přetvářejí v~konkrétní opatření ovlivňující život nás všech. V~rámci toho bylo snadné zapomenout, že věda sama umí pouze nabízet různé možnosti a scénáře, ale jejich reálné uvedení do praxe závisí na vládách. V~návaznosti na množství rozporuplných závěrů z~vědeckých studií však byla často v~rámci politiky situace interpretována na bázi selektivního výběru jen některých studií spíše než na bázi snahy o~syntézu dostupného poznání. To samozřejmě souvisí s~tím, že politik získává svůj kredit u~voličů více z~dojmu, který vyvolává. Vědec svůj kredit získává naopak na základě použitých metod a poznatků u~své komunity.

Není sice jisté, kdy a jak pandemie skončí, nicméně intenzivní vztah mezi vědou, politikou a médii pravděpodobně přetrvá. V~České republice by tyto tenze mohly být silným impulzem k~řadě reforem, které by ve finále mohly pomoci sblížit vědu a politiku. Inspiraci bychom mohli hledat třeba ve Velké Británii, kde působí vědecký poradní sbor vlády při případné nenadálé události. A~to jak v~pozitivním, tak negativním smyslu.

Vědě by tenze naopak mohly pomoci zlepšit způsob, jakým komunikuje o~tom, co dělá a k~čemu to vlastně celé je. Věda je součástí společnosti. Z~velké části je také společností placená. Je tedy v~zájmu vědců, aby dokázali dobře prezentovat své výsledky a vyvarovat se při tom toho, aby dělali jen to, o~čem umějí dobře komunikovat. To vše je potřebné, neméně tak je potřebné i to, aby vědci sami veřejně a srozumitelně popularizovali i to, jak rozhodování nad jejich daty a výzkumy změnilo společnost, a to nejen v~případě, kdy byli bráni v potaz, ale také v~možných scénářích, kdy se na jejich doporučení nedbalo. Tady se otevírá prostor ke vzniku skutečné politiky založené na faktech. Snad k~tomu přispěje i tento sborník.

